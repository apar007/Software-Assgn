\documentclass{article}

\title{Project Report: Random Audio Playlist}
\author{Apar Saxena (bt22btech11003)}
\date{\today}

\begin{document}
\maketitle

\section{Introduction}
The goal of this project was to develop a Python script that generates a random audio playlist and plays the audio files in a random order. The script allows users to listen to their collection of audio files in a fresh and unpredictable sequence, adding excitement to the listening experience.

\section{Implementation}
The implementation involved the following steps:

\begin{itemize}
    \item \textbf{Audio Directory:} The script begins by specifying the directory where the audio files are located. Users need to provide the actual path to the audio directory.
    \item \textbf{File Retrieval:} Using the \texttt{os.listdir} function, the script retrieves a list of all files within the specified audio directory. It then filters this list to include only the files with the '.mp3' extension.
    \item \textbf{Randomization:} The filtered list of audio files is shuffled using the \texttt{random.shuffle} function. This randomization step ensures that the playlist is unique and unpredictable for each run.
    \item \textbf{Audio Playback:} The script uses the \texttt{playsound} library to play each audio file in the shuffled list. The \texttt{playsound} function is called within a \texttt{for} loop, sequentially playing the audio files.
    
\end{itemize}

\section{Usage}
To use the script, follow these steps:

\begin{enumerate}
    \item Install the \texttt{playsound} library by running \texttt{pip install playsound} in the terminal or command prompt.
    \item Modify the \texttt{audio\_dir} variable in the script to specify the path to the directory containing your audio files.
    \item Run the script using Python: \texttt{python audio\_playlist.py}.
    
\end{enumerate}

\section{Conclusion}
The random audio playlist script provides a simple and effective solution for generating random playback sequences of audio files. By shuffling the playlist, users can enjoy their audio collection in a fresh and exciting way, adding variety and unpredictability to their listening experience.

\section{Further Enhancements}
The project can be further improved in the following ways:

\begin{itemize}
    \item Implement a graphical user interface (GUI) to provide a user-friendly interface for selecting the audio directory and controlling the playback.
    \item Support multiple audio file formats to cater to different user preferences.
    \item Allow users to customize the playlist length or specify the number of times the playlist should be played.
    \item Implement a feature to skip to the next audio file during playback.
    \item Enable the script to remember the last played position and resume from that point in the next session.
\end{itemize}

Overall, this project serves as a foundation for building more advanced audio playlist management systems and provides a starting point for adding additional features to enhance the user experience.

\end{document}
